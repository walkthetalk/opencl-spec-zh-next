%
% author:	Ni Qingliang
% date:		2011-02-11
%
\startcomponent c_glo

\chapter[chapter:glossary]{術語}

\startclglo[app][應用程式][Application]
運行在\refglo{host}和 OpenCL \refglo{device}上的程式。
\stopclglo

\startclglo[acquire][捕獲語義][Acquire semantics]
爲同步操作定義的一種內存順序語義,是一種原子操作,由內存中裝載。
假設有兩個執行單元, A 和 B,共享一個原子對象 M;
如果 A 要用捕獲語義原子裝載 M,而 B 要用釋放語義存儲 M,
則 A 的原子裝載要先發生,然後 B 才能看到 A 所進行的後續操作。
注意:內存順序 release、 sequentially consistent 和 acquire_release 都包含釋放語義,
可以與捕獲語義成對使用。
\stopclglo

\startclglo[acqrel][捕獲釋放語義][Acquire release semantics]
同步操作(如原子操作)所用的一種內存順序語義,即包含捕獲內存的順序,也包括釋放內存的順序。
主要用於讀-改-寫(RMW)操作。
\stopclglo

\startclglo[atomicop][原子操作][Atomic operations]
在任何時候、任何地點,要麼執行完,要麼不執行。
原子操作會限制存取的可見性。
\stopclglo

\startclglo[beac][阻塞的入隊 API 調用][Blocking Enqueue API calls]
參見\refglo{nbeac}。
\stopclglo

\startclglo[nbeac][非阻塞的入隊 API 調用][Non-Blocking Enqueue API calls]
\refglo{nbeac}在\refglo{cmdq}中放置一個\refglo{cmd}後會立刻返回到\refglo{host}中。
而\refglo{beac}則會等到\refglo{cmd}結束後再返回。
\stopclglo

% 如果刪除行 A,而將 B 改為空行,則編譯失敗,不知原因
\startclglo[barrier][屏障][Barrier]
有兩種\refglo{barrier}——\refglo{cmdqbarrier}和\refglo{workgrpbarrier}。
% placeholder A
\startigBase
\item \refglo{cmdqbarrier}本身是一個\refglo{cmd},
OpenCL API 提供有函式可以將其入隊。
他可以保證在此之前入隊的所有\refglo{cmd}都執行完畢後,
在此之後入隊的\refglo{cmd}才會開始執行。
% placeholder B
\item OpenCL C 編程語言中有一個內建的\refglo{workgrpbarrier}函式。
\refglo{kernel}在\refglo{device}上運行時,
可以利用此函式在同一\refglo{workgrp}中的不同\refglo{workitem}間進行同步
(這些\refglo{workitem}正在執行此\refglo{kernel})。
對這個\refglo{workgrp}而言,
其中任一\refglo{workitem}要想越過此\refglo{barrier}繼續執行,
則他所包含的所有\refglo{workitem}都必須先執行此\refglo{barrier}。
\stopigBase
\stopclglo

\startclglo[bufobj][緩衝對象][Buffer Object]
一種\refglo{memobj},其中存儲的是線性字節序列。
\refglo{kernel}在\refglo{device}上運行時,可以通過指位器來存取他。
\refglo{host}可以調用 OpenCL API 來操控\refglo{bufobj}。
\refglo{bufobj}封裝了以下資訊:
\startigBase
\item 字節數;
\item 一些屬性,用來描述使用情況以及分配自哪個區域;
\item 緩衝數據。
\stopigBase
\stopclglo

\startclglo[bikernel][內建內核][Built-in Kernel]
一種\refglo{kernel},
可以在 OpenCL \refglo{device}上運行,
也可以在具有固定功能的\refglo{customdev}上運行,
甚至可以在固件中運行。
\refglo{app}可以查詢\refglo{device}或\refglo{customdev}所支持的\refglo{bikernel}。
\refglo{programobj}可以包含用 OpenCL C 寫就的\refglo{kernel},
也可以包含\refglo{bikernel},
但不能同時包含兩者。
請參考\refglo{kernel}和\refglo{program}。
\stopclglo

\startclglo[cmd][命令][Command]
即 OpenCL 運算,可以提交給\refglo{cmdq}執行。
例如,一些 OpenCL \refglo{cmd}可以在計算設備上執行\refglo{kernel}、
操控\refglo{memobj}等等。
\stopclglo

\startclglo[cmdq][命令隊列][Command-queue]
這種對象中持有\refglo{cmd},而這些\refglo{cmd}將在某個特定\refglo{device}上執行。
\refglo{cmdq}是在\refglo{context}中的某個特定\refglo{device}上創建的。
\refglo{cmd}按序入隊,但不一定按順序執行。
參見\refglo{inordexec}和\refglo{outordexec}。
\stopclglo

\startclglo[cmdqbarrier][命令隊列屏障][Command-queue Barrier]
參見\refglo{barrier}。
\stopclglo

\startclglo[computedevmem][計算設備內存][Compute Device Memory]
附着到計算設備上的一塊或多塊內存。
\stopclglo

\startclglo[computeunit][計算器件][Compute Unit]
一個 OpenCL \refglo{device}中包含一個或多個\refglo{computeunit}。
一個\refglo{workgrp}只能在單個\refglo{computeunit}上執行。
一個\refglo{computeunit}由一個或多個\refglo{prcele}以及\refglo{locmem}組成。
\refglo{computeunit}也可能包含專用的材質濾波器件,此器件可以被他的\refglo{prcele}存取。
\stopclglo

\startclglo[concurrency][並發][Concurrency]
系統的一個屬性,這樣的系統中,多個任務可以同時保持活躍並取得進展。
運行\refglo{program}時要想並發執行,程式員必須找出問題中可以並發的部分,
並在源碼中標示出來,用一個支持並發的符號進行開發。
\stopclglo

\startclglo[constmem][不變內存][Constant Memory]
\refglo{glbmem}中的一個區域,在\refglo{kernel}執行過程中保持不變。
放入其中的\refglo{memobj}都是由\refglo{host}分配並初始化的。
\stopclglo

\startclglo[context][上下文][Context]
\refglo{kernel}在執行時所處的環境,同步和內存管理就是在此概念上定義的。
\refglo{context}包括一組\refglo{device}、這些\refglo{device}可以存取的內存、
相應的內存屬性、以及一個或多個\refglo{cmdq}
(用來對\refglo{kernel}的執行和\refglo{memobj}上的相關運算進行調度)。
\stopclglo

\startclglo[customdev][自定義設備][Custom Device]
一種 OpenCL \refglo{device},完整地實現了 OpenCL 運行時,
但不支持用 OpenCL C 編寫的\refglo{program}。
\refglo{customdev}可能是不可編程的專用硬件,高效節能,僅可執行定向任務;
也可能具有有限的編程能力,如專用的 DSP。
\refglo{customdev}並不符合 OpenCL。
\refglo{customdev}可能支持在線編譯器。
可以使用 OopenCL 運行時 API,
由源碼(如果支持在線編譯器)和/或二元碼、或者\refglo{bikernel}
來創建針對這種設備的\refglo{program}。
請參考\refglo{device}。
\stopclglo

\startclglo[dppm][數據並行編程模型][Data Parallel Programming Model]
傳統意義上,這種編程模型中,並發由單個\refglo{program}中的指令表示,
這些指令會作用到一組數據結構中的多個元素上。
在 OpenCL 中,對這個術語進行了推廣,此模型中,
單個\refglo{program}中的一組指令同時作用到一個索引抽象域中的所有點上。
\stopclglo

\startclglo[device][设备][Device]
\refglo{device}是一組\refglo{computeunit}的集合。
\refglo{cmdq}可用來將\refglo{cmd}在\refglo{device}上進行排隊。
這些\refglo{cmd}包括執行\refglo{kernel}或讀寫\refglo{memobj}。
典型的 OpenCL \refglo{device}指 GPU、
多核 CPU 以及其他處理器(如 DSP 和 Cell/B.E. 處理器)。
\stopclglo

\startclglo[evtobj][事件對象][Event Object]
\refglo{evtobj}封裝了運算(如一個\refglo{cmd})的狀態。
他可以用來對\refglo{context}中的運算進行同步。
\stopclglo

\startclglo[evtwaitlist][事件等待序列][Event Wait List]
\refglo{evtwaitlist}是\refglo{evtobj}的序列,可用來控制何時執行\refglo{cmd}。
\stopclglo

\startclglo[framework][框架][Framework]
一個軟件系統,包含一套用以支持軟件開發和執行的組件。
典型的\refglo{framework}包括庫、 API、運行時系統、編譯器等等。
\stopclglo

\startclglo[glbid][全局 ID][Global ID]
\refglo{glbid}用來唯一標識一個\refglo{workitem},
源自執行\refglo{kernel}時所指定的全局\refglo{workitem}數目。
\refglo{glbid}是一個 N 維的值,起自 \math{(0, 0, \cdots 0)}。
請參考\refglo{locid}。
\stopclglo

\startclglo[glbmem][全局內存][Global Memory]
一塊\refglo{memregion},\refglo{context}中的所有\refglo{workitem}均可存取。
\refglo{host}可以使用\refglo{cmd}來存取他(如讀、寫以及映射)。
\stopclglo

\startclglo[glsharegrp][GL 共享組][GL share group]
這種對象用來管理共享的 OpenGL 或 OpenGL ES 資源(如材質、緩衝、幀緩衝或渲染緩衝),
並與一個或多個 GL 上下文對象關聯在一起。
典型的\refglo{glsharegrp}是不透明的,不能直接存取。
\stopclglo

\startclglo[handle][句柄][Handle]
一種不透明的型別,用來引用 OpenCL 所分配的\refglo{obj}。
對\refglo{obj}的任何操作都是通過引用其\refglo{handle}來進行的。
\stopclglo

\startclglo[host][主機][host]
\refglo{host}使用 OpenCL API 與\refglo{context}進行交互。
\stopclglo

\startclglo[hostptr][主機指位器][Host pointer]
一種指位器,指向\refglo{host}上虛擬位址空間中的內存。
\stopclglo

\startclglo[illegal][違規][Illegal]
明確規定不允許的系統行為,遇到這種問題時, OpenCL 會將其視為錯誤進行報告。
\stopclglo

\startclglo[imgobj][圖像對象][Image Object]
一種\refglo{memobj},用來存儲二維或三維的結構化陣列。
只能通過讀寫函式存取其中的圖像數據。讀取函式需要使用\refglo{sampler}。

\refglo{imgobj}封裝了以下資訊:
\startigBase
\item 圖像的維數;
\item 圖像中每個元素的描述;
\item 一些屬性,用來描述使用情況以及分配自哪個區域;
\item 圖像數據。
\stopigBase

圖像中的所有元素均從預定義的一系列圖像格式中選取。
\stopclglo

\startclglo[impdef][依賴於具體實作][Implementation Defined]
在符合 OpenCL 標準的不同實作中,明確規定可以不同的那些行為。
對於這種行為,要求 OpenCL 的實作者提供相關文檔。
\stopclglo

\startclglo[inordexec][順序執行][In-order Execution]
OpenCL 中的一種執行模型,\refglo{cmdq}中的\refglo{cmd}按提交的順序執行,
只有當正在運行的\refglo{cmd}完成後,下一個才能開始運行。
参看\refglo{outordexec}。
\stopclglo

\startclglo[kernel][內核][Kernel]
\refglo{program}中聲明的一種函式,
可以在 OpenCL \refglo{device}上執行。
在\refglo{program}中定義的任何函式,
都可以通過加上限定符 \cqlf{__kernel} 或 \cqlf{kernel} 將其標識成\refglo{kernel}。
\stopclglo

\startclglo[kernelobj][內核對象][Kernel Object]
\refglo{kernelobj}封裝了
\refglo{program}中聲明的一個 \cqlf{__kernel} 函式
以及執行此函式時所需的引數。
\stopclglo

\startclglo[locid][局部 ID][Local ID]
\refglo{workitem}的一種 ID,在所處\refglo{workgrp}中是唯一的。
\refglo{locid}是一個 N 維的值,起自 \math{(0, 0, \cdots 0)}。
請參考\refglo{glbid}。
\stopclglo

\startclglo[locmem][局部內存][Local Memory]
一塊\refglo{memregion},隸屬於某個\refglo{workgrp},
只有這個\refglo{workgrp}中的\refglo{workitem}才能存取。
\stopclglo

\startclglo[marker][標註][Marker]
一種\refglo{cmd},可用來給之前入隊的所有\refglo{cmd}打標籤。
此\refglo{cmd}會返回一個事件,\refglo{app}可以等在這個事件上,
例如等待之前入隊的\refglo{cmd}全部完成。
\stopclglo

\startclglo[memobj][內存對象][Memory Object]
某塊帶有\refglo{refcnt}的\refglo{glbmem}的\refglo{handle}。
請參考\refglo{bufobj}和\refglo{imgobj}。
\stopclglo

\startclglo[memregion][內存區域][Memory Region]
OpenCL 中一個確切的位址空間。
不同的\refglo{memregion}在物理上可能重疊,但在邏輯上 OpenCL 會將其視為不同。
分為\refglo{prvmem}、\refglo{locmem}、\refglo{constmem}以及\refglo{glbmem}。
\stopclglo

\startclglo[mempool][內存池][Memory Pool]
同\refglo{memregion}。
\stopclglo

\startclglo[obj][對象][Object]
這是對 OpenCL API 可以操控的資源的一種抽象表示。
包括\refglo{programobj}、\refglo{kernelobj}以及\refglo{memobj}。
\stopclglo

\startclglo[outordexec][亂序執行][Out-of-Order Execution]
一種執行模型,\refglo{cmdq}中的\refglo{cmd}可以任意順序開始和結束執行,
只要符合\refglo{evtwaitlist}和\refglo{cmdqbarrier}的限制即可。
參見\refglo{inordexec}。
\stopclglo

\startclglo[pardev][父設備][Parent Device]
一種 OpenCL \refglo{device},被劃分成了多個\refglo{subdev}。
並不是所有的\refglo{pardev}都是\refglo{rootdev}。
\refglo{rootdev}可以被拆分,\refglo{subdev}可以繼續被拆分。
這種情況下,第一組\refglo{subdev}可能是第二組\refglo{subdev}的\refglo{pardev},
但不是\refglo{rootdev}。
請參考\refglo{device}、\refglo{subdev}和\refglo{rootdev}。
\stopclglo

\startclglo[platform][平台][platform]
包括\refglo{host}以及 OpenCL \refglo{framework}所管理的一些\refglo{device}。
\refglo{app}可以在這些\refglo{device}上共享\refglo{res}、執行\refglo{kernel}。
\stopclglo

\startclglo[prvmem][私有內存][Private Memory]
專屬某個\refglo{workitem}的一塊內存。
在一個\refglo{workitem}的\refglo{prvmem}中定義的變量
對另一個\refglo{workitem}是不可見的。
\stopclglo

\startclglo[prcele][處理元件][Processing Element]
一種虛擬的標量處理器。一個\refglo{workitem}可以在一個或多個\refglo{prcele}上執行。
\stopclglo

\startclglo[program][程式][Program]
由一組\refglo{kernel}組成,
可能還包含一些輔助函式(由 \cqlf{__kernel} 函式調用)和常量数据。
\stopclglo

\startclglo[programobj][程式對象][Program Object]
\refglo{programobj}封裝有以下資訊:
\startigBase
\item 對所屬\refglo{context}的引用。

\item \refglo{program}源碼或二元碼。

\item 最近成功構建的程式執行體,用來運行此\refglo{program}的\refglo{device}清單,
所用的構建選項以及構建日誌。

\item 當前附着其上的\refglo{kernelobj}的數目。
\stopigBase
\stopclglo

\startclglo[refcnt][引用計數][Reference Count]
OpenCL 對象的生命週期由其\refglo{refcnt}決定,
這個內部計數記錄了此對象被引用的次數。
在 OpenCL 中創建一個對象時,其\refglo{refcnt}被置為0。
后续对\refglo{retain} API
(如 \capi{clRetainContext}、 \capi{clRetainCommandQueue})
的調用會使其\refglo{refcnt}增一。
而對\refglo{release} API
 如 \capi{clReleaseContext}、 \capi{clReleaseCommandQueue})
的調用會使其\refglo{refcnt}減一。
當\refglo{refcnt}降為 0 後, OpenCL 會回收此對象所佔用的資源。
\stopclglo

\startclglo[relaxconsist][放寬的一致性][Relaxed Consistency]
一種內存一致性模型。
對同一塊內存而言,不同的\refglo{workitem}或\refglo{cmd}所看到的內容可能不同,
當然,\refglo{barrier}和其他顯式同步點除外。
\stopclglo

\startclglo[res][資源][Resource]
OpenCL 所定義的一類\refglo{obj}。
\refglo{res}的實體就是\refglo{obj}。
最常用的\refglo{res}是\refglo{context}、\refglo{cmdq}、\refglo{programobj}、
\refglo{kernelobj}以及\refglo{memobj}。
計算用資源主要指那些參與推進程式計數器動作的硬件元件。
例如\refglo{host}、\refglo{device}、\refglo{computeunit}以及\refglo{prcele}。
\stopclglo

\startclglo[retain][保留][Retain]
參見\refglo{release}。
\stopclglo

\startclglo[release][釋放][Release]
會使 OpenCL \refglo{obj}的\refglo{refcnt}增一( retain)或減一( release)。
這是一個記賬功能,
可以保證在使用某個\refglo{obj}的所有實體都結束之前,系統不會移除此\refglo{obj}。
參見\refglo{refcnt}。
\stopclglo

\startclglo[rootdev][根設備][Root Device]
一種 OpenCL \refglo{device},還未劃分。
請參考\refglo{device}、\refglo{pardev}和\refglo{subdev}。
\stopclglo

\startclglo[sampler][採樣器][Sampler]
一種\refglo{obj},用來描述\refglo{kernel}讀取圖像時怎樣對其採樣。
讀取圖像的函式將\refglo{sampler}作為一個引數。
\refglo{sampler}指定圖像的尋址方式,
如圖像坐標溢出時如何處置、濾波模式、輸入的圖像坐標是否已歸一化。
\stopclglo

\startclglo[simd][單指令多數據][SIMD]
一種編程模型,一個\refglo{kernel}在多個\refglo{prcele}上並發執行,
每個\refglo{prcele}上都有自己的數據,並共享一個程式計數器。
所有\refglo{prcele}所執行的指令嚴格一致。
\stopclglo

\startclglo[spmd][單程式多數據][SPMD]
一種编程模型,一個\refglo{kernel}在多個\refglo{prcele}上並發執行,
每個\refglo{prcele}上都有自己的數據和自己的程式計數器。
因此,運行同一個\refglo{kernel}的所有計算資源都會維護自己的指令計數器;
同時由於\refglo{kernel}中分支的存在,這些\refglo{prcele}中的實際指令序列可能會有很大不同。
\stopclglo

\startclglo[subdev][子設備][Sub Device]
一個 OpenCL \refglo{device}可以劃分成多個\refglo{subdev}。
按照劃分方案,新的\refglo{subdev}是\refglo{pardev}中一組\refglo{computeunit}的別名。
任何情況下,只要可以使用其\refglo{pardev},就可以使用這些\refglo{subdev}。
劃分\refglo{device}不會摧毀\refglo{pardev},可以繼續和其\refglo{subdev}混合使用。
請參考\refglo{device}、\refglo{pardev}和\refglo{rootdev}。
\stopclglo

\startclglo[tppm][任務並行編程模型][Task Parallel Programming Model]
一種編程模型,用多個並發的任務來表示計算,
其中任務就是單個\refglo{workgrp}(大小是1)中所執行的\refglo{kernel}。
這些並發的任務可以運行不同的\refglo{kernel}。
\stopclglo

\startclglo[thsafe][線程安全][Thread-safe]
當一個 OpenCL API 被多個\refglo{host}線程同時調用時,
如果 OpenCL 所管理的內部狀態始終保持一致,就認為他是\refglo{thsafe}的。
如果一个 OpenCL API 是\refglo{thsafe}的,
那麼就允許\refglo{app}在多個\refglo{host}線程中同時調用他,
而不必在這些線程間實施互斥,即他們也是重入安全的。
\stopclglo

\startclglo[undef][未定義][Undefined]
調用 OpenCL API、\refglo{kernel}中使用內建函式或者執行\refglo{kernel}時,
如果 OpenCL 沒有對其行為顯式定義,那麼這種行為就是\refglo{undef}的。
至於在 OpenCL 中碰到這種情況時會發生什麼,不要求實作明確指出。
\stopclglo

\startclglo[workgrp][作業組][Work-group]
一組相關的\refglo{workitem},在同一個\refglo{computeunit}上執行。
其中所有\refglo{workitem}執行同一個\refglo{kernel},
並共享\refglo{locmem}和\refglo{workgrpbarrier}。
\stopclglo

\startclglo[workgrpbarrier][作業组屏障][Work-group Barrier]
參見\refglo{barrier}。
\stopclglo

\startclglo[workitem][作業項][Work-item]
\refglo{kernel}的並行執行體中的一個。
作為在\refglo{computeunit}上執行的\refglo{workgrp}中的一部分,
一個\refglo{workitem}可以由一個或多個\refglo{prcele}執行。
用其\refglo{glbid}和\refglo{locid}來區分\refglo{workitem}。
\stopclglo

% OpenCL class diagram
\section{OpenCL 类图}

\useURL[uml][http://www.uml.org/]
\reffig{classdiag}使用統一建模語言\footnote{%
統一建模語言(\from[uml])是 Object Management Group (OMG)的商標。}
(UML)以類圖的方式對 OpenCL 規範進行了描述。
圖中的節點和邊分別代表類以及類之間的關係。
簡化起見,類中沒有特性和操作。
抽象類帶有注釋“\{abstract\}”。
圖中顯示了類之間的關係,有聚集(aggregation,帶有實心方塊)、
關聯(association,没有註解)和繼承(inheritance,帶有開放的箭頭)。
邊兩端帶有關係的基數。
其中“{\ftEmp{*}}”代表“多個”,“{\ftEmp{1}}”代表“一個且只有一個”,
“{\ftEmp{0..1}}”代表“可選的一個”,而“{\ftEmp{1..*}}”代表“一個或多個”。
普通箭頭代表關係的方向。

\placefigure[here][fig:classdiag]
{OpenCL 類圖}{\externalfigure[figclassdiag-1.pdf]}

\stopcomponent

