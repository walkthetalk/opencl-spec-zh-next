% 首行缩进
\setupindenting[first,always,2em]
% 段落间距
\setupwhitespace[small] %none small medium big line fixed fix dimension
% 行间距
\setupinterlinespace[medium]
%\setupinterlinespace[big, on][line=3ex]
%\setuprelativeinterlinespace[1.2]
% setupalign 必須位於setupbodyfont之前,因此修改了zhfonts模塊
%\setupalign[flushleft,nothyphenated,broad]
%width left right middle inner outer wide broad height bottom line reset hanging
%nothanging hyphenated nothyphenated lesshyphenation morehyphenation new old normal yes
%no flushleft flushright flushouter flushinner center hz nohz spacing nospacing tolerant
%verytolerant stretch

% code
%\installprettytype[C][C]
\setuplinenumbering[
% conversion	numbers characters Characters romannumerals Romannumerals text
% start	number
% step	number
%  width=1em,	%dimension
%  location=text,	%intext inmargin
% style	normal bold slanted boldslanted type cap small... command
% prefix	text
% referencing	on off
]
\definetextbackground[verb][
  frame=off,
  location=paragraph,
  leftoffset=0.5em,
  rightoffset=0.5em,
  topoffset=0.25em,bottomoffset=0.25em,
  rulethickness=0.75pt,
  strut=no,
  backgroundcolor={gray:1},
]

\usemodule[t-pretty-clfunc]
\setuptyping[CLFUNC][
  margin=no,	%dimension standard yes no
  escape=yes,	%character %{[[,]]}
  tab=8,	%number yes no
  numbering=no,	%line file no
]

\setuptyping[
  escape=yes,	%character %{[[,]]}
  %space	on off
  tab=8,	%number yes no
]

\definetyping[clc][%option=c,
  bodyfont=9pt,
  %space	on off
  %page	yes no
  %option	slanted normal commands color none
  %text	yes no
  %icommand	command
  %vcommand	command
  %ccommand	command
  before={\vskip.5ex\starttextbackground[verb]},	%command
  after={\stoptextbackground\vskip.5ex},	%command
  margin=no,	%dimension standard yes no
  %evenmargin	dimension
  %oddmargin	dimension
  %blank	dimension small medium big standard halfline line
  escape=yes,	%character %{[[,]]}
  %space	on off
  tab=8,	%number yes no
  %page	yes no
  %indentnext	yes no
  style=\tt\tf,		%normal bold slanted boldslanted type cap small... command
  %color	name
  %palet	name
  %lines=yes,	%yes no hyphenated
  %empty	yes all no
  numbering=line,	%line file no
]

% c in table
\definetyping[cintbl][
  option=none,
  before={\blank},
  after={\blank},
  tab=8,
]

% note
\defineframedtext[notepar][
  bodyfont=small,	%5pt ... 12pt small big
  style=\ftRef, %normal bold slanted boldslanted type cap small... COMMAND
  %left=,	%COMMAND
  %right,	%COMMAND
  %before=,	%COMMAND
  %after,	%COMMAND
  %inner,	%COMMAND
  %linecorrection=,	%on off
  %depthcorrection=,	%on off
  %margin=,	%standard yes no
  %location=,	%left right middle none
  indenting={first,always,2em},	%never none not no yes always first next small medium big normal odd even DIMENSION
  %inherits from \setupframed
  background=screen,
  frame=off,
  rightframe=on,
  leftframe=on,
  framecolor=darkgreen,
  rulethickness=3pt,
  width=local,
]

% note
\defineframedtext[replacepar][
  bodyfont=small,	%5pt ... 12pt small big
%  style=\ftRef, %normal bold slanted boldslanted type cap small... COMMAND
  %left=,	%COMMAND
  %right,	%COMMAND
  %before=,	%COMMAND
  %after,	%COMMAND
  %inner,	%COMMAND
  %linecorrection=,	%on off
  %depthcorrection=,	%on off
  %margin=,	%standard yes no
  %location=,	%left right middle none
  indenting={first,always,2em},	%never none not no yes always first next small medium big normal odd even DIMENSION
  %inherits from \setupframed
  background=screen,
  frame=off,
  rightframe=on,
  leftframe=on,
  framecolor=darkgreen,
  rulethickness=3pt,
  width=local,
]

%%%%%%%%%%%%%%%%%%%%%%%%%%%%%%% english %%%%%%%%%%%%%%%%%%%%%%%%%%%%%%%%%%%%%%%%
% 在 definestartstop 中使用 setupindenting 等會修改全局設定,因此改用 \def
\def\startEnglish{\dostartEnglish}

\long\def\dostartEnglish#1\stopEnglish{%
{\setscript[no]
\mainlanguage[en]
\setupindenting[no]%
\setupwhitespace[line]%
\setupinterlinespace[medium]%
#1}
}

\def\startEnglishWithoutWhiteSpace{\dostartEnglishWithoutWhiteSpace}

\long\def\dostartEnglishWithoutWhiteSpace#1\stopEnglishWithoutWhiteSpace{%
{\setscript[no]
\mainlanguage[en]
\setupindenting[no]%
\setupwhitespace[medium]%
\setupinterlinespace[medium]%
\blank[1ex]
#1%
\blank[1ex]}
}
