\setupTABLE[%
    %frameoffset=.5\linewidth,
    %backgroundoffset=\v!frame,
    %framecolor=\s!black,
    %width=\v!fit,
    %height=\v!fit,
    %autowidth=\v!yes,
    %rulethickness=\linewidth,
    %strut=\v!yes,
    %autostrut=\v!no,
    %
    %color=,
    style={\rmx},
    headstyle={\rmx\bf},
    %headcolor=,
    %aligncharacter=\v!no,
    %alignmentcharacter={,},
    %option=, % \v!stretch
    %header=,
    %spaceinbetween=,
    %maxwidth=8em,
    %textwidth=\hsize,
    split=yes,	% repeat
    %splitoffset=0pt,
    %distance=\zeropoint,           % individual column
    %columndistance=\zeropoint,     % each column (whole table)
    %leftmargindistance=\zeropoint, % whole table
    %rightmargindistance=\zeropoint,% whole table
    %left=,
    %right=,
    %setups=,
    splitmethod=b%
]
\setupTABLE[header][
    style={\rmx\bf},
    background=color,
    backgroundcolor=gray:1,
]

%%%%%%%%%%%%%%%%%%% ETD: enum / type / desc
\def\startETD{\dodoubleempty\dostartETD}

\long\def\dostartETD[#1][#2]#3\stopETD{%
  \startlocalfootnotes
  \bTABLE[option=stretch]
  \bTABLEhead
    \bTR[background=color,backgroundcolor=gray:1]
        \bTH \ctype{#1} \eTH
        \bTH #2 \eTH
    \eTR
  \eTABLEhead
  \bTABLEbody
    #3
  \eTABLEbody
  \eTABLE
  \placelocalfootnotes[style=\rmx,before=,after=,]
  \stoplocalfootnotes
}

\define[3]\clETDMulti{
\bTR[background=color,backgroundcolor=gray:1]
  \bTC #1 \eTC
  \bTC \ctype{#2} \eTC
\eTR
\bTR
  \bTC[nc=2]
    \parindent2em
    #3
  \eTC
\eTR
}

\define[3]\clETD{
\bTR[background=color,backgroundcolor=gray:1]
  \bTC \cenum{#1} \eTC
  \bTC \ctype{#2} \eTC
\eTR
\bTR
  \bTC[nc=2]
    \parindent2em
    #3
  \eTC
\eTR
}
%%%%%%%%%%%%%%%%%%% ED: enum / desc
\def\startED{\dodoubleempty\dostartED}

\long\def\dostartED[#1]#2\stopED{%
  \startlocalfootnotes
  \bTABLE[option=stretch]
  \bTABLEhead
    \bTR[background=color,backgroundcolor=gray:1]
        \bTH #1 \eTH
    \eTR
  \eTABLEhead
  \bTABLEbody
    #2
  \eTABLEbody
  \eTABLE
  \placelocalfootnotes[style=\rmx,before=,after=,]
  \stoplocalfootnotes
}

\define[2]\clED{
\bTR[background=color,backgroundcolor=gray:1]
  \bTC \cenum{#1} \eTC
\eTR
\bTR
  \bTC
    \parindent2em
    #2
  \eTC
\eTR
}
%%%%%%%%%%%%%%%%%%% OD: object / desc
\def\startCLOD{\dodoubleempty\dostartCLOD}

\long\def\dostartCLOD[#1][#2]#3\stopCLOD{%
  \startlocalfootnotes
  \bTABLE[option=stretch]
  \bTABLEhead
    \bTR[background=color,backgroundcolor=gray:1]
        \bTH #1 \eTH
        \bTH #2 \eTH
    \eTR
  \eTABLEhead
  \bTABLEbody
    #3
  \eTABLEbody
  \eTABLE
  \placelocalfootnotes[style=\rmx,before=,after=,]
  \stoplocalfootnotes
}

\define[2]\clOD{
\bTR
  \bTD #1 \eTD
  \bTD
    \parindent2em
    #2 \eTD
\eTR
}
\define[2]\clMD{
\clOD{\cmacro{#1}}{#2}
}
%%%%%%%%%%%%%%%%%%% OO: object / object
\def\startCLOO{\dodoubleargument\dostartCLOO}

\long\def\dostartCLOO[#1][#2]#3\stopCLOO{%
  \midaligned{
  \bTABLE
  \setupTABLE[c][each][align={middle,lohi}]
  \bTABLEhead
    \bTR[background=color,backgroundcolor=gray:1]
        \bTH #1 \eTH
        \bTH #2 \eTH
    \eTR
  \eTABLEhead
  \bTABLEbody
    #3
  \eTABLEbody
  \eTABLE
  }
}

\define[2]\clOO{
\bTR
  \bTD #1 \eTD
  \bTD #2 \eTD
\eTR
}

% macro - macro
\define[1]\clMM{
\clOO{\cmacroemp{#1}}{\cmacroemp{CL_#1}}
}

% macro - macro for half
\define[1]\clMMH{
\clMM{HALF_#1}
}

% macro - macro for float
\define[1]\clMMF{
\clMM{FLT_#1}
}

% macro - macro for double
\define[1]\clMMD{
\clMM{DBL_#1}
}

% constant - math
\define[2]\clCM{
\clOO{\cmacroemp{#1}}{\math{#2}}
}
%%%%%%%%%%%%%%%%%%% FD: fuction / desc
\def\startCLFD{\dostartCLFD}

\long\def\dostartCLFD#1\stopCLFD{%
  \startlocalfootnotes
  \bTABLE[option=stretch]
  \setuptyping[option=CLFUNC]

  \bTABLEhead
    \bTR[background=color,backgroundcolor=gray:1]
        \bTH 函式 \eTH
        \bTH 描述 \eTH
    \eTR
  \eTABLEhead
  \bTABLEbody
    #1
  \eTABLEbody
  \eTABLE
  \placelocalfootnotes[style=\rmx,before=,after=,]
  \stoplocalfootnotes
}

\define[1]\clFD{
\bTR
  \bTD \typebuffer[funcproto:#1] \eTD
  \bTD \parindent2em\getbuffer[funcdesc:#1] \eTD
\eTR
}
%%%%%%%%%%%%%%%%%%% FD: fuction / accuracy
\def\startCLFA{\dostartCLFA}

\long\def\dostartCLFA[#1][#2]#3\stopCLFA{%
  \bTABLE
  \setupTABLE[c][odd][align={flushright,lohi}]
  \setupTABLE[c][even][align={flushleft,lohi}]
  \bTABLEhead
    \bTR[background=color,backgroundcolor=gray:1]
        \bTH #1 \eTH
        \bTH #2 \eTH
    \eTR
  \eTABLEhead
  \bTABLEbody
    #3
  \eTABLEbody
  \eTABLE
}

\define[2]\clFA{
\bTR
  \bTD #1 \eTD
  \bTD #2 \eTD
\eTR
}

% api
\define[2]\clFAA{
\bTR
  \bTD \capi{#1} \eTD
  \bTD #2 \eTD
\eTR
}

% math
\define[2]\clFAM{
\bTR
  \bTD \math{#1} \eTD
  \bTD #2 \eTD
\eTR
}
